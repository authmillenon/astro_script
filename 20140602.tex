\section{Interstellare Verfärbung}
\paragraph{Ursache} Interstellare Materie / Speziell: Staubteilchen

% TODO Graphic

\begin{itemize}
    \item Re-Emission bei $T_{\text{Staub}} \sim 10^2\,\kelvin \rightarrow$ IR ("`Rötung"')
    \item isotrope Streuung (kleine Staubteilchen) $\rightarrow$ aus der Richtung gestreut
\end{itemize}

\paragraph{Verfärbung} Schwächung der scheinbaren Helligkeit
\[ \Delta m_{\lambda} = A_{\lambda} \sim \frac{1}{\lambda} \]

\paragraph{Beschreibung} Farbexzess $E$
\[ E_{B-V} = \underset{\text{beobachtet}}{(B-V)_{\mathrm{obs}}} - \underbrace{(B-V)_0}_{\text{Eigenfarbe des Objekts}} = A_B - A_V \]

Empirischer Zusammenhang: $A_V \approx (3.0 \pm 0.2)\,E_{B-V}$

\section{Bolometrische Korrektur (B.C.)}
! Sterne !

% TODO Graphic

\[ \boxed{B.C. = m_v - m_{\mathrm{bol}}} \]

\framebox{Korrekturen werden aus Sternatmosphärenmodellen abgeleitet.}

\section{Susammenhang: Absolute bolometrische Helligkeit $\leftrightarrow$ Leuchtkraft}
\[ L_* = 4 \pi R_*^2 F_* \]

\paragraph{Annahme} keine Verfärbung

\begin{align*}
    r = 10\,\parsec{:}\ f(10\,\parsec) &= F_*\left(\frac{R_*}{10\,\parsec}\right)^2 = \frac{L_*}{4 \pi \cancel{R_*^2}} \left(\frac{\cancel{R_*}}{10\,\parsec}\right)^2 \\
    m_{\mathrm{bol}} (10\,\parsec) = M_{\mathrm{bol}}^{*} &= -2.5 \log f (10\,\parsec) + C \\
                                                          &= -2.5 \log \left(\frac{f_* R_*^2}{(10\,\parsec)^2}\right) \\
                                                          &= -2.5 \log (L_*) + 2.5 \cancel{\log(4\pi)} + \cancel{5} \underset{\approx 1}{\cancel{\log(10\,\parsec)}} + \cancel{C}
\end{align*}

Analog für Sonne:
\begin{align*}
    M_{\mathrm{bol}}^{\astrosun} &= -2.5 \log L_{\astrosun} + 2.5 \cancel{\log(4\pi)} + \cancel{5} + \cancel{C} \\
    \Rightarrow M_{\mathrm{bol}}^* - M_{\mathrm{bol}}^{\astrosun} &= -2.5 \log\left(\frac{L_*}{L_{\astrosun}}\right) \\
    M_{\mathrm{bol}}^{\astrosun} &= 4.72\magnitude \\
    M_{\mathrm{bol}}^* &= 4.72\magnitude - 2.5\magnitude \log\left(\frac{L_*}{L_{\astrosun}}\right)
\end{align*}

\section{Spektralklassifikation und HR-Diagramm}
\paragraph{Prinzip} Spektren werden in Klassen zusammengefasst

\paragraph{Umrechnung nach Havard-Sequenz} $\rightarrow T_{\mathrm{eff}}$

\[ O - B - A - F - G - K - \overbracket{M - L - T - Y}^{\mathclap{\text{Braune Zwerge}}} \]

% TODO HRD

\paragraph{Problem} HRD ist bezüglich Spektralklassen nicht eindeutig $\rightarrow$ Leuchtkraftklassen

\subsubsection{Masse-Leuchtkraftbezeichnung auf der Hauptreihe}
\[ \log\left(\frac{L_*}{L_{\astrosun}}\right) \sim \log \left(\frac{M_*}{M_{\astrosun}}\right) \]

\[
    \boxed{\log\left(\frac{L_*}{L_{\astrosun}}\right) = \begin{cases}
        2 \log \left(\frac{M_*}{M_{\astrosun}}\right) - 0.4, & \text{sonst} \\
        4 \log \left(\frac{M_*}{M_{\astrosun}}\right), & \text{für $M_* > 0.6 M_{\astrosun}$} \\
    \end{cases}}
\]

% TODO Graph

\subsubsection{Fundamentale Sternparameter}

\[ \underbrace{\underbrace{L_*, T_*}_{R_*}, M_*}_{\text{Oberflächenbeschleunigung $g_* = \dfrac{GM_*}{R_*^2}$}}, \underset{\text{dynamische Elemente (Elementzusammensetzung -- Metallizität)}}{\{\epsilon_i\}} \]

% next chapter

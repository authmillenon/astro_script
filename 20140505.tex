\section{2-Körperproblem -- Kepler Gesetze}
\paragraph{Skizze:}
\begin{center}
    \tdplotsetrotatedcoords{0}{0}{90}
    \begin{tikzpicture}[scale=2,tdplot_rotated_coords]
        \coordinate (O) at (0,0,0);
        \coordinate [label=left:{$m_1$}] (m1) at (-1,1,-0.2);
        \coordinate [label=right:{$m_2$}] (m2) at (1,1.2,0.4);
        \coordinate (S) at (0,1.1,0.1);

        \draw [thick,->] (O) -- (1.2,0,0) node [anchor=north west] {$x$};
        \draw [thick,->] (O) -- (0,1.2,0) node [anchor=south west] {$y$};
        \draw [thick,->] (O) -- (0,0,1.2) node [anchor=south] {$z$};

        \draw [->] (O) -- (m1) node [pos=0.5,anchor=north east] {$\vec r_1$};
        \draw [->] (O) -- (m2) node [pos=0.5,anchor=north west] {$\vec r_2$};
        \draw [->] (m1) -- (m2) node [pos=0.5, anchor=south] {$\vec r$};
        \draw [->] (O) -- (S) node [pos=0.5, anchor=east] {$R$};
    \end{tikzpicture}
\end{center}

\paragraph{DGL}
\begin{align*}
    m_1 \vec r_1 + m_2 \vec r_2 = \frac{d^2}{dt^2} (m_1 \vec r_1 + m_2 \vec r_2) &= \vec 0 \tag*{$| \dfrac{m}{m}$} \\
    \Rightarrow m \frac{d^2}{dt^2} \underbrace{\left(\frac{m_1\vec r_1 + m_2 \vec r_2}{m}\right)}_{\mathclap{= \vec R = \frac{\vec C_2}{m} \text{für $\vec C_1 = \vec P_{\mathrm{ges}}$}}} = \vec 0 \Leftrightarrow m \ddot{\vec{R}} &= \vec 0
\end{align*}
\[ \boxed{m\ddot{\vec{R}} = \vec 0} \]
\framebox{$\Rightarrow$ Referezsystem bzgl. Massenschwerpunkt muss ein Inertialsystem sein}

$\Rightarrow$ \textbf{Wahl:} Koordinatenursprung im Schwerpunkt $\Leftrightarrow \vec{R} = \vec 0$
